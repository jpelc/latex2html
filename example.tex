\documentclass[10pt,a4paper]{article}
\usepackage{polski}
\usepackage[utf8]{inputenc}
\usepackage{textgreek}
\usepackage{textcomp}
\usepackage[margin=1in]{geometry}

\title{Konwerter LaTeX2HTML}
\author{Jakub Pelc, Tomasz Korecki}
\date{2014}

\begin{document}

\maketitle
%w dokumencie możliwe jest używanie jednoliniowych komentarzy

\section{Wstęp}
Ten dokument stanowi prezentację możliwości stworzonego przez nas konwertera. W poszczególnych rozdziałach i podrozdziałach przedstawione zostaną kolejne konstrukcje, które obsługiwane są przez aplikację.\par
Ten tekst stanowi nowy akapit. W celu jego uzyskania zastosowana została komenda ``par'', która w tym tekście umieszczona została w cudzysłowie (który jest kolejną specjalną konstrukcją).\newline O proszę, nowa linia.

\section{Obsługiwane konstrukcje}
W tym rozdziale przedstawione zostaną konstrukcje obsługiwane przez nasz konwerter.

\subsection{Operacje na tekście}
W tym podrozdziale przedstawione zostaną sposoby modyfikacji tekstu.

\subsubsection{Emph}
Tekst może być standardowy. Jednak możliwe jest, aby \emph{był wyróżniony}. Nie są to jednak jedyne możliwości jego modyfikacji. W dalszej części zostanie to udowodnione.

\subsubsection{Underline}
Dotarliśmy do dowodu. Tekst może też być \underline{podkreślony}.

\subsubsection{Bold}
Ale to jeszcze nie wszystko. Tekst możemy także \textbf{pogrubić}.
 
\subsection{Znaki specjalne}
W tekście możemy stosować wiele znaków specjalnych. Oto one. Albo nie. Teraz jeszcze napiszę kilka słów, a znaki specjalne przedstwione zostaną w kolejnym paragrafie --- są bardzo ważne.\par
Znaki specjalne: - -- --- \# \$ \% \^{} \& \_ \{ \} \~{} \par
Oto znaki specjalne (z klamrami na końcu): \slash{} \textbackslash{} \ldots{} \textcelsius{} \texteuro{} \textGamma{} \textDelta{} \textTheta{} \textLambda{} \textPi{} \textSigma{} \textPhi{} \textPsi{} \textOmega{} \textalpha{} \textbeta{} \textgamma{} \textdelta{} \textepsilon{} \texteta{} \texttheta{} \textiota{} \textkappa{} \textlambda{} \textmugreek{} \textnu{} \textxi{} \textpi{} \textrho{} \textsigma{} \texttau{} \textupsilon{} \textphi{} \textchi{} \textpsi{} \textomega{}. \par
Today wstawia datę, więc umieścimy tą komendę w nowym paragrafie: \today{}.

\subsection{Środowiska}
W tym podrozdziale przedstawione zostaną wszystkie obsługiwane przez omawiany konwerter środowiska (zdziwieni? Konwerter jest naprawdę zaawansowany).

\subsubsection{Itemize}
Poniżej lista punktowana. Możliwości jest kilka:
\begin{itemize}
\item Punkt pierwszy
\item Kolejny
\item Następny \textbf{pogrubiony}, ładne pogrubienie
\item A ten \emph{wyróżniony},  ładne wyróżnienie
\item A ten z kolei \underline{podkreślony}, ładne podkreślenie
\end{itemize}
A to nowy paragraf po liście punktowanej.

\subsubsection{Enumerate}
Poniżej lista numerowana. Możliwości jest kilka:
\begin{enumerate}
\item Punkt pierwszy
\item Kolejny
\item Następny \textbf{pogrubiony}, ładne pogrubienie
\item A ten \emph{wyróżniony}, ładne wyróżnienie
\item A ten z kolei \underline{podkreślony}, ładne podkreślenie
\end{enumerate}
A to nowy paragraf po liście numerowanej.

\subsubsection{Flushleft}
Tutaj przedstawiam blok tekstu wyrównany do lewej:
\begin{flushleft}
Ten tekst jest wyrównany do lewej strony. Czary\ldots{} Tutaj dzieją się czary.
\end{flushleft}
A to nowy paragraf po bloku wyrównanym do lewej.

\subsubsection{Center}
Tutaj przedstawiam blok tekstu wyśrodkowany:
\begin{center}
Ten tekst jest wyśrodkowany. Czary\ldots{} Tutaj dzieją się czary.
\end{center}
A to nowy paragraf po bloku wyśrodkowanym.

\subsubsection{Flushright}
Tutaj przedstawiam blok tekstu wyrównany do prawej:
\begin{flushright}
Ten tekst jest wyrównany do prawej strony. Czary\ldots{} Tutaj dzieją się czary.
\end{flushright}
A to nowy paragraf po bloku wyrównanym do prawej.

\subsubsection{Quote}
Tutaj przedstawiam blok tekstu cytowanego:
\begin{quote}
Ten tekst jest cytowany. Czary\ldots{} Tutaj dzieją się czary.
\end{quote}
A to nowy paragraf po bloku cytowanym.

\subsubsection{Verbatim}
Tutaj przedstawiam blok tekstu, który zostanie umieszczony w dokumencie w sposób dosłowny:
\begin{verbatim}
Tekst dużo spacji       dużo, prawda?
\end{verbatim}
A to nowy paragraf po bloku verbatim.

\subsubsection{Tabular}
Poniżej przykładowa tabela: \newline
\begin{tabular}{|c|c|c|}
	\hline
	\emph{Kolumna 1} & Kolumna 2 & Kolumna 3 \\ \hline
	Wiersz 2 & Wiersz 2 & \textbf{Wiersz 2} \\ \hline
	Wiersz 3 & \underline{Wiersz 3} & Wiersz 3 \\
	\hline
\end{tabular}
\newline
Koniec tabeli. To już nowy akapit.


\section{Zakończenie}
Dokument ten jest dowodem na to, iż założenia zadania projektowego zostały w pełni zrealizowane. Mam nadzieję, że wszystkie elementy zostały w nim zawarte. Z wyjątkiem sytuacji, w których konieczne jest użycie nieobsługiwanych przez nasz konwerter wyrażeń matematycznych, może okazać się on naprawdę przydatny (oczywiście do podstawowych zastosowań). Jako autorzy jesteśmy zadowoleni z osiągniętego rezultatu. Z pewnościa warto byłoby zająć się rozwojem tego narzędzia (pozbycie się ograniczeń, obsługa wyrażeń matematycznych, obsługa bardziej zaawansowanych tabel, etc.).

\end{document}

